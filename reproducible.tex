\documentclass[12pt,a4paper]{article}

\usepackage{anysize}
\marginsize{2cm}{2cm}{2cm}{2cm}
\usepackage{graphicx,amsfonts,amssymb,bm}

\begin{document}
\thispagestyle{empty}
\begin{center}
  \begin{Large}
    How to reproduce results from
  \end{Large}

  \begin{large}
    \emph{Spatial Point Patterns: Methodology and Applications with R}\\
    Baddeley, Rubak and Turner
  \end{large}
\end{center}

Readers who run the example code in the book
will find that some of their results are
now different from the results shown in the book.

This is caused by recent changes to the \texttt{spatstat} package
and the \textsf{R} system.

The authors try to ensure, wherever possible, that statements in the book
remain true, and \textsf{R} code in the book remains valid, 
using the current version of \texttt{spatstat}.

However, a few changes are unavoidable. Here is a list of actions that
the user can take to ensure that the results are reproducible.

\begin{enumerate}
\item The \texttt{spatstat} package has now been split into several
  sub-packages. In order to execute all the code in the book,
  you may need to type
\begin{verbatim}
    library(spatstat)
    library(spatstat.utils)
    library(spatstat.gui)
\end{verbatim}
\item Data files that were in \texttt{spatstat}
  are now in the sub-package \texttt{spatstat.data}.
  If you use the command \texttt{system.file} to extract data files,
  use the argument \texttt{package="spatstat.data"}.
\item 
  Use the following commands to ensure that simulations
  (and simulation-based calculations and figures) in the book
  are exactly reproducible using the chosen values of the random seed.
  They select the older, slower
  simulation algorithms that were in force when the book was written.
\begin{verbatim}
    RNGkind(sample.kind="Rounding")
    spatstat.options(fastthin=FALSE)
    spatstat.options(fastpois=FALSE)
\end{verbatim}
  These should only be used to reproduce the output in the book.
  Please be warned that \texttt{RNGkind(sample.kind="Rounding")} is undesirable
  for new work, because it produces poor samples in large populations.
\item
  The function \texttt{lengths.psp} has been renamed \verb!lengths_psp!
  to avoid a conflict with the new generic function \texttt{lengths}
  in \textsf{R}.
\item
  The new package \texttt{local} for local likelihood, mentioned
  in Sections~9.13, 12.5 and 13.11, has been renamed \texttt{spatstat.local}.
  It is available on CRAN.
\item
  Check the list of errata on \texttt{book.spatstat.org}.
\end{enumerate}

\vspace*{\fill}


\hspace*{\fill} \today

\end{document}
